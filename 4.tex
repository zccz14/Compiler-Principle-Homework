\begin{enumerate}
    \item[1.] 给出下面表达式的逆波兰表示(后缀式):
    
    \begin{enumerate}
        \item a * (-b + c)
        \item not A or not (C or not D)
        \item a + b * (c + d / e)
        \item (A and B) or (not C or D)
        \item -a + b * (-c + d)
        \item (A or B) and (C or not D and E)
        \item if (x + y) * z = 0 then (a + b) ↑ c else a ↑ b ↑ c
    \end{enumerate}
    
    \textbf{答:}
    
    用 neg 表示一元运算符取负数 -,以便区分减法运算符。
    
    \begin{enumerate}
        \item a b neg c + *
        \item A not C D not or not or
        \item a b c d e / + * +
        \item A B and C not D or or
        \item a neg b c neg d + * +
        \item A B or C D not E and or and
        \item x y + z * 0 = a b + c ↑ a b ↑ c ↑ ?
    \end{enumerate}
    
    \item[3.] 请将表达式 -(a + b) * (c + d) - (a + b + c) 分别表示成三元式、间接三元式和四元式序列。
    
    \textbf{答:}

    用 neg 表示一元运算符取负数 -,以便区分减法运算符。
    
    \textbf{三元式}:
        
    \begin{tabular}{cccc}
        (1) & + & a & b \\
        (2) & neg & (1) & _ \\
        (3) & + & c & d \\
        (4) & * & (2) & (3) \\
        (5) & + & a & b \\
        (6) & + & (5) & c \\
        (7) & - & (4) & (6)
    \end{tabular}
    
    \textbf{间接三元式}:
    
    观察到上表中 (1) 与 (5) 是重复的,可以使用间接三元式来消除冗余三元式:
    
    间接码表(横排):(1) (2) (3) (4) (1) (5) (6)
    
    间接三元式表:

    \begin{tabular}{cccc}
        (1) & + & a & b \\
        (2) & neg & (1) & _ \\
        (3) & + & c & d \\
        (4) & * & (2) & (3) \\
        (5) & + & (1) & c \\
        (6) & - & (4) & (5)
    \end{tabular}
    
    \textbf{四元式}:
    
    \begin{tabular}{ccccc}
        (1) & + & a & b & T1 \\
        (2) & neg & T1 & _ & T2 \\
        (3) & + & c & d & T3 \\
        (4) & * & T2 & T3 & T4 \\
        (5) & + & a & b & T5 \\
        (6) & + & T5 & c & T6 \\
        (7) & - & T4 & T6 & T7 \\
    \end{tabular}
    
    \item[5.] 按上课所讲方式写出四元式序列:
    
    \begin{center}
        A[i, j] := B[i, j] + C[A[k, l] + D[i + j]]
    \end{center}
    
    其中,数组 A, B 的规模均为 20 × 40,数组 C, D 的长度均为 80。
    
    \textbf{答:}
    
    \begin{tabular}{ccccc}
        (1) & * & i & 40 & T1 \\
        (2) & + & T1 & j & T2 \\
        (3) & - & A & 41 & T3 \\

        (4) & * & i & 40 & T4 \\
        (5) & + & T4 & j & T5 \\
        (6) & - & B & 41 & T6 \\
        (7) & =[] & T6[T5] & _ & T7 \\

        (8) & * & k & 40 & T8 \\
        (9) & + & T8 & l & T9 \\
        (10) & - & A & 41 & T10 \\
        (11) & =[] & T10[T9] & _ & T11 \\

        (12) & + & i & j & T12 \\
        (13) & - & D & 1 & T13 \\
        (14) & =[] & T13[T12] & _ & T14 \\
        
        (15) & + & T11 & T14 & T16 \\
        (16) & - & C & 1 & T16 \\
        (17) & =[] & T16[T15] & _ & T17 \\

        (18) & + & T7 & T17 & T18 \\
        (19) & []= & T18 & _ & T3[T2] \\
    \end{tabular}
    
    \item[6.] 按上课所讲方式,写出布尔式 A or (B and not (C or D)) 的四元式序列。
    
    \textbf{答:}
    
    上课所讲方式包含两种方式:
    
    \begin{enumerate}
        \item 将布尔表达式当作表达式直接翻译

        \begin{tabular}{ccccc}
            (1) & or & C & D & T1 \\
            (2) & not & T1 & _ & T2 \\
            (3) & and & B & T2 & T3 \\
            (4) & or & A & T3 & T4 \\
        \end{tabular}
        
        \item 拉链返填短路优化布尔表达式
        
        \begin{tabular}{ccccc}
            (1) & jnz & A & _ & 0\\
            (2) & j & _ & _ & 3 \\
            (3) & jnz & B & _ & 5 \\
            (4) & j & _ & _ & 0 \\
            (5) & jnz & C & _ & 4 \\
            (6) & j & _ & _ & 7 \\
            (7) & jnz & D & _ & 5\\
            (8) & j & _ & _ & 1 \\
        \end{tabular}
        
    \end{enumerate}
    
    
    \item[7.] 按上课所讲方式,把下面的语句翻译成四元式序列:
    
    \begin{lstlisting}
    while A < C and B < D do
        if A = 1 then C := C + 1 else
            while A <= D do A := A + 2;
    \end{lstlisting}
    
    \textbf{答:}
    
    \begin{tabular}{ccccc}
        (1) & jlt & A & C & 3 \\
        (2) & j & _ & _ & 0 \\
        (3) & jlt & B & D & 5 \\
        (4) & j & _ & _ & 2 \\
        (5) & je & A & 1 & 7 \\
        (6) & j & _ & _ & 10 \\
        (7) & + & C & 1 & T1 \\
        (8) & := & T1 & _ & C \\
        (9) & j & _ & _ & 1 \\
        (10) & jle & A & D & 12 \\
        (11) & j & _ & _ & 1 \\
        (12) & + & A & 2 & T2 \\
        (13) & := & T2 & _ & A \\
        (14) & j & _ & _ & 10 \\
        (15) & j & _ & _ & 1 \\
    \end{tabular}
    
\end{enumerate}