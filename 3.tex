\begin{enumerate}
    \item[1.] 令文法 $G_1$ 为:
    
    \begin{align*}
        E & \to E + T | T \\
        T & \to T * F | F \\
        F & \to (E) | i \\
    \end{align*}
    
    证明 $ E + T * F $ 是它的一个句型,指出这个句型的所有短语,直接短语和句柄。
    
    \textbf{答:}
    
    从起始符号 E 开始推导:
    
    \begin{align*}
        E & \Rightarrow E + T \\
          & \Rightarrow E + T * F
    \end{align*}
    
    因此 $E + T * F$ 是文法 $G_1$ 的一个句型,\textbf{证毕}。
    
    短语:$E + T * F$, $T * F$
    
    直接短语:$T * F$
    
    句柄:$T * F$
    
    \item[3.] 令文法\footnotemark $G_2$ 为:
    
    \footnotetext{此处文法 $G_2$ 来自于练习题 2,但练习题 2 不要求完成,因此改动了题目消除了对练习题 2 的依赖。}
    
    \begin{align*}
        S & \to a | \wedge | (T) \\
        T & \to T, S | S 
    \end{align*}
    
    \begin{enumerate}
        \item 计算文法 $G_2$ 的 FIRSTVT 和 LASTVT。
        
        \textbf{答:}
        
        $$
            S \to a \Rightarrow \left\{
                \begin{array}{rl}
                    a & \in FIRSTVT(S) \\
                    a & \in LASTVT(S) 
                \end{array}
            \right.
        $$
        
        $$
            S \to \wedge \Rightarrow \left\{
                \begin{array}{rl}
                    \wedge & \in FIRSTVT(S)  \\
                    \wedge & \in LASTVT(S)
                \end{array}
            \right.
        $$
        
        $$
            S \to (T) \Rightarrow \left\{
                \begin{array}{rl}
                    ( & \in FIRSTVT(S) \\
                    ) & \in LASTVT(S)
                \end{array}
            \right.
        $$
        
        $$
            T \to T, S \Rightarrow \left\{
                \begin{array}{rl}
                    FIRSTVT(T) & \subseteq FIRSTVT(T) \\
                    LASTVT(S) & \subseteq LASTVT(T) \\
                    , & \in FIRSTVT(T) \\
                    , & \in LASTVT(T)
                \end{array}
            \right.
        $$
        
        $$
            T \to S \Rightarrow \left\{
                \begin{array}{rl}
                    FIRSTVT(S) & \subseteq FIRSTVT(T) \\
                    LASTVT(S) & \subseteq LASTVT(T) 
                \end{array}
            \right.
        $$
        
        综上关系解得:
        
        \begin{table}[H]
            \centering
            \begin{tabular}{|c|c|c|}
                $V_N$ & FIRSTVT & LASTVT \\
                S & \{a, $\wedge, ($\} & \{a, $\wedge$, )\} \\
                T & \{a, $\wedge, (, \textbf{,}$\} & \{a, $\wedge$, ), \textbf{,}\} 
            \end{tabular}
            \caption{文法 $G_1$ 的 FIRSTVT 和 LASTVT}
            \label{tab:FLVT_1}
        \end{table}
        
        
        \item 计算 $G_2$ 的优先关系。$G_2$ 是一个算符优先文法吗?
        
        \textbf{答:}
        
        \begin{enumerate}
            \item \textbf{对于 $G_2$ 的任意产生式的右部不含两个相继的非终结符,因此 $G_2$ 是算符文法;}
            
            \item \textbf{$G_2$ 不含 $\epsilon$ 产生式;}
            
            \item 优先级关系:
            
            $$
                S \to (T) \Rightarrow \left\{
                    \begin{array}{rll}
                        ( & \lessdot t & , \forall t \in FIRSTVT(T) \\
                        t & \gtrdot ) & , \forall t \in LASTVT(T) \\
                        ( & \eqcirc )
                    \end{array}
                \right.
            $$
            
            $$
                T \to T, S \Rightarrow \left\{
                    \begin{array}{rll}
                        t & \gtrdot , & , \forall t \in LASTVT(T) \\
                        , & \lessdot t & , \forall t \in FIRSTVT(S)
                    \end{array}
                \right.
            $$
            
            对于 $\#S\#$ 有:
            
            $$
                \left\{
                    \begin{array}{rll}
                        \# & \lessdot t & , \forall t \in FIRSTVT(T) \\
                        t & \gtrdot \# & , \forall t \in LASTVT(T) \\
                        \# & \eqcirc \# 
                    \end{array}
                \right.
            $$
            
            综上关系得到优先关系表:
            
            \begin{table}[H]
                \centering
                \begin{equation*}
                    \begin{array}{c|cccccc}
                          & a & \wedge & ( & ) & , & \# \\
                        \hline
                        a & & & & \gtrdot & \gtrdot & \gtrdot \\
                        \wedge & & & & \gtrdot & \gtrdot & \gtrdot \\
                        ( & \lessdot & \lessdot & \lessdot & \eqcirc & \lessdot & \\
                        ) & & & & \gtrdot & \gtrdot & \gtrdot \\
                        , & \lessdot & \lessdot & \lessdot & \gtrdot & \gtrdot & \\
                        \# & \lessdot & \lessdot & \lessdot & & & \eqcirc 
                    \end{array}
                \end{equation*}
                \caption{文法 $G_2$ 的优先关系表}
                \label{tab:PT_3}
            \end{table}
            
        
            \textbf{没有任何优先级冲突。}
            
        \end{enumerate}
        
        综上所述,文法 $G_2$ 是算符优先文法。
        
        \item 计算 $G_2$ 的优先函数。
        
        \textbf{答:}
        
        \textbf{等价地给出一种基于优先关系表的算法,避免使用箭线图。}
        
        首先将算符优先文法的优先关系表 $T$ 定义为:$T: (V_T \cup \{\#\})^2 \mapsto \{\{\lessdot\}, \{\gtrdot\}, \{\eqcirc\}, \emptyset \} $。
        
        步骤分为两个阶段:\textbf{等价类划分}与\textbf{拓扑排序}。先根据 $\eqcirc$ 关系对元素进行等价类划分;然后将 $T$ 看成一个二分图的邻接矩阵的形式,进行拓扑排序,标记节点的拓扑序号。
        
        \begin{enumerate}
            \item 初始化关系表 $T := T$,计数器 $cnt := 1$
            
            \item 列出元素集 $\{f_a, g_a | a \in V_T \cup \{\#\}\}$

            \item 根据 $\eqcirc$ 关系划分等价类。

            \item 对于每个等价类,检查其中的元素,是否\textbf{全部}满足以下条件:
            
            \begin{enumerate}
                \item 若是行元素(有 f 标志)
                
                $T'$ 中对应的行中不存在 $\gtrdot$
                
                \item 若是列元素(有 g 标志)
                
                $T'$ 中对应的列中不存在 $\lessdot$
                
            \end{enumerate}

            如果是,将该等价类标记为当前计数器的值,并在 $T'$ 中过滤该等价类有关的行与列,得到余表 $remain(T')$。
            
            \item 若 $T' = remain(T')$,意味着 $T$ 中存在环,即不存在优先函数,算法结束。
            
            \item 更新表 $T' := remain(T')$。
            
            \item 若 $T'$ \textbf{作为图的意义下}非空,则计数器自增 1 ,跳到步骤 4。
            
            \item 否则清算每个等价类中的元素的优先级,元素的优先级等于其所在的等价类被标记的值。
            
            \item 算法结束。

        \end{enumerate}
        
        对于文法 $G_2$,参考表 \ref{tab:PT_3} 运行算法:
        
        \begin{enumerate}
            \item 初始化 $T' := T$ ,计数器 $cnt := 1$。
            
            \item 根据表 \ref{tab:PT_3},先得到元素集为 $\{f_a, f_\wedge, f_(, f_), f_,, f_\#, g_a, g_\wedge, g_(, g_), g_,, g_\#\}$。
            
            \item 根据 $\eqcirc$ 关系划分等价类:$\{f_a\}, \{f_\wedge\}, \{f_)\}, \{f_,\}, \{g_a\}, \{g_\wedge\}, \{g_(\}, \{g_,\}, \{f_(, g_)\}, \{f_\#, g_\#\}$。
            
            \item $\{f_\#, g_\#\}, \{f_(, g_)\}$ 符合要求,标记为 1,并划去对应的行与列得到余表:

            \begin{equation*}
                \begin{array}{c|cccc}
                      & a & \wedge & ( & , \\
                    \hline
                    a & & &  & \gtrdot \\
                    \wedge & & & & \gtrdot \\
                    ) & & & & \gtrdot \\
                    , & \lessdot & \lessdot & \lessdot & \gtrdot \\
                \end{array}
            \end{equation*}
            
            \item 一轮拓扑排序结束,$T'$ 非空,计数器增加到 2。
            
            \item $\{g_,\}$ 符合条件,标记为 2,并划去对应的行列得到余表:
            
            \begin{equation*}
                \begin{array}{c|ccc}
                      & a & \wedge & ( \\
                    \hline
                    a & & & \\
                    \wedge & & & \\
                    ) & & & \\
                    , & \lessdot & \lessdot & \lessdot \\
                \end{array}
            \end{equation*}
            
            \item $T'$ 非空,计数器增加到 3。
            
            \item $\{f_a\}, \{f_\wedge\}, \{f_)\}, \{f_,\}$ 符合要求,标记为 3,求余表:
            
            \begin{equation*}
                \begin{array}{c|ccc}
                      & a & \wedge & ( \\
                    \hline
                \end{array}
            \end{equation*}
            
            \item 表非空,计数器增加到 4。
            
            \item $\{g_a\}, \{g_\wedge\}, \{g_(\}$ 符合要求,标记为 4,求余表为空。
            
            \item 得到的优先函数表如下:
            
            \begin{table}[H]
                \centering
                \begin{equation*}
                    \begin{array}{c|cc}
                         & f & g \\
                         \hline
                         a & 3 & 4 \\
                         \wedge & 3 & 4 \\
                         ( & 1 & 4 \\
                         ) & 3 & 1 \\
                         , & 3 & 2 \\
                         \# & 1 & 1
                    \end{array}
                \end{equation*}
                \caption{文法 $G_2$ 的优先函数表}
                \label{tab:PF_3_1}
            \end{table}
            
            \item 算法结束。
            
        \end{enumerate}
        
        对比箭线图计数算法的结果:
        
        \begin{table}[H]
            \centering
            \begin{equation*}
                \begin{array}{c|cc}
                     & f & g \\
                     \hline
                     a & 3 & 4 \\
                     \wedge & 3 & 4 \\
                     ( & 1 & 4 \\
                     ) & 3 & 1 \\
                     , & 3 & 2 \\
                     \# & 1 & 1
                \end{array}
            \end{equation*}
            \caption{文法 $G_2$ 的优先函数表(通过箭线图计可达组数)}
            \label{tab:PF_3_2}
        \end{table}

        \begin{table}[H]
            \centering
            \begin{equation*}
                \begin{array}{c|cc}
                     & f & g \\
                     \hline
                     a & 5 & 6 \\
                     \wedge & 5 & 6 \\
                     ( & 2 & 6 \\
                     ) & 5 & 2 \\
                     , & 5 & 3 \\
                     \# & 2 & 2
                \end{array}
            \end{equation*}
            \caption{文法 $G_2$ 的优先函数表(通过箭线图计可达点数)}
            \label{tab:PF_3_3}
        \end{table}
        
        \textbf{优先函数的等价性}:如果对于两种优先函数$F_{p1}, F_{p2}$,能找到一个单调增函数$F$,对于任意优先函数定义域内的元素 $x$ 使得 $F(F_{p1}(x)) = F_{p2}(x)$ 成立,则表示 $F_{p1}, F_{p2}$ 等价。
        
        对比表 \ref{tab:PF_3_1},\ref{tab:PF_3_2},\ref{tab:PF_3_3} 容易找到对应的单增函数表明这三者是等价的。
        
        \item \sout{给出输入串 (a, (a, a)) 的算符优先分析过程。}
    \end{enumerate}
    
    \item[5.] 考虑文法
    
    \begin{align*}
        S & \to AS | b \\
        A & \to SA | a
    \end{align*}
    
    \begin{enumerate}
            
        \item 列出这个文法的所有 LR(0) 项目。
        
        \textbf{答:}
        
        \begin{enumerate}[(1)]
            \item $S \to \cdot AS$
            \item $S \to A \cdot S$
            \item $S \to AS \cdot$
            \item $S \to \cdot b$
            \item $S \to b \cdot$
            \item $A \to \cdot SA$
            \item $A \to S \cdot A$
            \item $A \to SA \cdot$
            \item $A \to \cdot a$
            \item $A \to a \cdot$
        \end{enumerate}
        
        \item 构造这个文法的 LR(0) 项目集规范族及识别活前缀的 DFA。
        
        拓广文法:

        \begin{enumerate}[(1)]
            \item $S' \to S$
            \item $S \to AS$
            \item $S \to b$
            \item $A \to SA$
            \item $A \to a$
        \end{enumerate}
        
        子集法构造 LR(0) 项目集规范族:
        
        \begin{equation*}
            \begin{array}{cl}
                I_0 = & \{S' \to \cdot S, S \to \cdot AS, S \to \cdot b, A \to \cdot SA, A \to \cdot a \} \\
                \hline
                I_1 = & GO(I_0, S) = \{S' \to S \cdot, A \to S \cdot A, A \to \cdot SA, A \to \cdot a, S \to \cdot AS, S \to \cdot b\} \\
                I_2 = & GO(I_0, A) = \{S \to A \cdot S, S \to \cdot AS, S \to \cdot b, A \to \cdot SA, A \to \cdot a \} \\
                I_3 = & GO(I_0, b) = \{S \to b \cdot \} \\
                I_4 = & GO(I_0, a) = \{A \to a \cdot \} \\
                \hline
                I_5 = & GO(I_1, A) = \{A \to SA \cdot, S \to A \cdot S, S \to \cdot AS, S \to \cdot b, A \to \cdot SA, A \to \cdot a \} \\
                I_6 = & GO(I_1, S) = \{A \to S \cdot A, A \to \cdot SA, A \to \cdot a, S \to \cdot AS, S \to \cdot b\} \\
                    & GO(I_1, a) = \{A \to a \cdot \} = I_4 \\
                    & GO(I_1, b) = \{S \to b \cdot\} = I_3 \\
                \hline
                I_7 = & GO(I_2, S) = \{S \to AS \cdot, A \to S \cdot A, A \to \cdot SA, A \to \cdot a, S \to \cdot AS, S \to \cdot b \} \\
                    & GO(I_2, A) = \{S \to A \cdot S, S \to \cdot AS, S \to \cdot b, A \to \cdot SA, A \to \cdot a \} = I_2 \\
                    & GO(I_2, b) = \{S \to b \cdot \} = I_3 \\
                    & GO(I_2, a) = \{S \to a \cdot \} = I_4 \\
                \hline
                    & GO(I_5, S) = \{S \to AS \cdot, A \to S \cdot A, A \to \cdot SA, A \to \cdot a, S \to \cdot AS, S \to \cdot b\} = I_7 \\
                    & GO(I_5, A) = \{S \to A \cdot S, S \to \cdot AS, S \to \cdot b, A \to \cdot SA, A \to \cdot a\} = I_2 \\
                    & GO(I_5, b) = \{S \to b \cdot\} = I_3 \\
                    & GO(I_5, a) = \{A \to a \cdot\} = I_4 \\
                \hline
                    & GO(I_6, A) = \{A \to SA \cdot, S \to A \cdot S, S \to \cdot AS, S \to \cdot b, A \to \cdot SA, A \to \cdot a \} = I_5 \\
                    & GO(I_6, S) = \{A \to S \cdot A, A \to \cdot SA, A \to \cdot a, S \to \cdot AS, S \to \cdot b \} = I_6 \\
                    & GO(I_6, a) = \{A \to a \cdot\} = I_4 \\
                    & GO(I_6, b) = \{S \to b \cdot\} = I_3 \\
                \hline
                    & GO(I_7, A) = \{A \to SA \cdot, S \to A \cdot S, S \to \cdot AS, S \to \cdot b, A \to \cdot SA, A \to \cdot a\} = I_5 \\
                    & GO(I_7, S) = \{A \to S \cdot A, A \to \cdot SA, A \to \cdot a, S \to \cdot AS, S \to \cdot b\} = I_6 \\
                    & GO(I_7, a) = \{A \to a \cdot\} = I_4 \\
                    & GO(I_7, b) = \{S \to b \cdot\} = I_3 \\
            \end{array}
        \end{equation*}
        
        LR(0) 项目集规范族为 $\{I_i | i = 0, 1, \dots 7\}$
        
        构造 DFA:
        
        \begin{figure}[H]
            \centering
            \begin{tikzpicture}[->,>=stealth',shorten >=1pt,auto,node distance=2.8cm,
                                semithick]
                \node[initial, state] (0) {0};
                \node[state] (2) [right of = 0] {2};
                \node[state] (5) [below left of = 2] {5};
                \node[state] (6) [below right of = 5] {6};
                \node[state] (7) [below right of = 2] {7};
                \node[state] (1) [left of = 6] {1};
                
                \path
                (0) edge node [left] {S} (1)
                    edge node {A} (2)
                (1) edge node {A} (5)
                    edge node {S} (6)
                (2) edge [loop above] node {A} (2)
                    edge node {S} (7)
                (5) edge [bend left] node {S} (7)
                    edge node {A} (2)
                (6) edge node {A} (5)
                    edge [loop below] node {S} (6)
                (7) edge node {S} (6)
                    edge [bend left] node {A} (5)
                ;
            \end{tikzpicture}
            \caption{识别活前缀的DFA \\ \textbf{注:图中所有节点(0, 1, 2, 5, 6, 7) 都有指向 3 的 b 边与指向 4 的 a 边,此处略去不画。}}
            \label{fig:DFA_5}
        \end{figure}
        
        \item 这个文法是 SLR 的吗?若是,构造出它的 SLR 分析表。
        
        \textbf{答:不是。}
        
        构造 LR(0) 分析表:
        
        \begin{equation*}
            \begin{array}{c|ccc|cc}
                I & a & b & \# & S & A \\
                \hline
                0 & s_4 & s_3 & & 1 & 2 \\
                1 & s_4 & s_3 & acc & 6 & 5 \\
                2 & s_4 & s_3 & & 7 & 2 \\
                3 & r_3 & r_3 & r_3 & & \\
                4 & r_5 & r_5 & r_5 & & \\
                5 & r_4, s_4 & r_4, s_3 & r_4 & 7 & 2 \\
                6 & s_4 & s_3 & & 6 & 5 \\
                7 & r_2, s_4 & r_2, s_3 & r_2 & 6 & 5\\
            \end{array}
        \end{equation*}
        
        有冲突,因为 $FOLLOW(S) \cap FOLLOW(A) = \{a, b\}$,SLR 法不能解决冲突,因此\textbf{此文法不是 SLR 的}。
        
        \item \sout{这个文法是 LALR 或 LR(1) 的吗?}
    \end{enumerate}
    
    \item[7.] 证明下面文法是 SLR(1) 但不是 LR(0) 的。
    
    \begin{align*}
        S & \to A \\
        A & \to Ab | bBa \\
        B & \to aAc | a | aAb
    \end{align*}
    
    \textbf{证:}
    
    文法的开始符号只有一个产生式所以不需要拓广文法。
    
    \begin{enumerate}[(1)]
        \item $S \to A$
        \item $A \to Ab$
        \item $A \to bBa$
        \item $B \to aAc$
        \item $B \to a$
        \item $B \to aAb$
    \end{enumerate}
    
    LR(0) 项目集规范族:
    
    \begin{equation*}
        \begin{array}{cl}
            I_0 = & \{S \to \cdot A, A \to \cdot Ab, A \to \cdot bBa\} \\
            \hline
            I_1 = & GO(I_0, A) = \{S \to A \cdot, A \to A \cdot b \} \\
            I_2 = & GO(I_0, b) = \{A \to b \cdot Ba, B \to \cdot aAc, B \to \cdot a, B \to \cdot aAb \}\\
            \hline
            I_3 = & GO(I_1, b) = \{A \to Ab \cdot\} \\
            \hline
            I_4 = & GO(I_2, B) = \{A \to bB \cdot a\} \\
            I_5 = & GO(I_2, a) = \{B \to a \cdot Ac, B \to a \cdot, B \to a \cdot Ab, A \to \cdot Ab, A \to \cdot bBa\} \\
            \hline
            I_6 = & GO(I_4, a) = \{A \to bBa \cdot \} \\
            \hline
            I_7 = & GO(I_5, A) = \{B \to aA \cdot c, B \to aA \cdot b, A \to A \cdot b \} \\
                  & GO(I_5, b) = \{A \to b \cdot Ba, B \to \cdot aAc, B \to \cdot a, B \to \cdot aAb \} = I_2 \\
            \hline
            I_8 = & GO(I_7, c) = \{B \to aAc \cdot\} \\
            I_9 = & GO(I_7, b) = \{B \to aAb \cdot, A \to Ab \cdot\}
        \end{array}
    \end{equation*}
    
    构造 LR(0) 分析表:
    
    \begin{table}[H]
        \centering
        \begin{equation*}
            \begin{array}{c|cccc|ccc}
                I & a & b & c & \# & S & A & B \\
                \hline
                0 & & s_2 & & & & 1 & \\
                1 & & s_3 & & acc & & & \\
                2 & s_5 & & & & & & 4 \\
                3 & r_2 & r_2 & r_2 & r_2 & & & \\
                4 & s_6 & & & & & & \\
                5 & r_5 & r_5, s_8 & r_5 & r_5 & & 7 & \\
                6 & r_3 & r_3 & r_3 & r_3 & & & \\
                7 & & s_{10} & s_9 & & & & \\
                8 & r_4 & r_4 & r_4 & r_4 & & & \\
                9 & r_6, r_2 & r_6, r_2 & r_6, r_2 & r_6, r_2 & & & \\
            \end{array}
        \end{equation*}
        \caption{具有冲突的分析表}
        \label{tab:SLR_7_1}
    \end{table}
    
    表中有许多冲突,这表明\textbf{该文法不是 LR(0) 的}。
    
    计算 FIRST 和 FOLLOW 集:
    
    \begin{table}[H]
        \centering
        \begin{tabular}{c|cc}
            $V_N$ & FIRST & FOLLOW \\
            \hline
            S & \{b\} & \{\#\} \\
            A & \{b\} & \{\#, b, c\} \\
            B & \{a\} & \{a\} \\
        \end{tabular}
        \caption{文法的 FIRST 和 FOLLOW 集}
        \label{tab:FF_7}
    \end{table}
    
    对于表 \ref{tab:SLR_7_1} 中 $I_5, I_{10}$ 中的冲突,应用 SLR(1) 法解决冲突:
    
    \begin{equation*}
        \begin{array}{c|cccc|ccc}
            I & a & b & c & \# & S & A & B \\
            \hline
            0 & & s_2 & & & & 1 & \\
            1 & & s_3 & & acc & & & \\
            2 & s_5 & & & & & & 4 \\
            3 & r_2 & r_2 & r_2 & r_2 & & & \\
            4 & s_6 & & & & & & \\
            5 & r_5 & s_8 & r_5 & r_5 & & 7 & \\
            6 & r_3 & r_3 & r_3 & r_3 & & & \\
            7 & & s_{10} & s_9 & & & & \\
            8 & r_4 & r_4 & r_4 & r_4 & & & \\
            9 & r_6 & r_2 & r_2 & r_2 & & & \\
        \end{array}
    \end{equation*}
    
    解决了所有冲突,这表明\textbf{该文法是 SLR(1) 的}。
    
    综上所述,该文法是 SLR(1) 的但不是 LR(0) 的,\textbf{证毕}。
    
    \item[8.] 证明下面的文法
    
    \begin{align*}
        S & \to AaAb | BbBa \\
        A & \to \epsilon \\
        B & \to \epsilon
    \end{align*}
    
    是 LL(1) 的但不是 SLR(1) 的。
    
    \textbf{证:}
    
    计算该文法的 FIRST 和 FOLLOW:
    
    \begin{table}[H]
        \centering
        \begin{tabular}{c|cc}
            $V_N$ & FIRST & FOLLOW \\
            \hline
            S & \{a, b\} & \{\#\} \\
            A & \{$\epsilon$\} & \{a, b\} \\
            B & \{$\epsilon$\} & \{a, b\} \\
        \end{tabular}
        \caption{文法的 FIRST 和 FOLLOW 集}
        \label{tab:FF_8}
    \end{table}
    
    对于 $S \to AaAb | BbBa $
    
    $ FIRST(AaAb) \cap FIRST(BbBa) = \{a\} \cap \{b\} = \emptyset $
    
    所以,\textbf{该文法是 LL(1) 的}。
    
    拓广文法:
    
    \begin{enumerate}[(1)]
        \item $S' \to S$
        \item $S \to AaAb$
        \item $S \to BbBa$
        \item $A \to \epsilon$
        \item $B \to \epsilon$
    \end{enumerate}
    
    LR(0) 项目集规范族:
    
    \begin{equation*}
        \begin{array}{cl}
            I_0 = & \{S' \to \cdot S, S \to \cdot AaAb, S \to \cdot BbBa, A \to \cdot, B \to \cdot \} \\
            \hline
            I_1 = & GO(I_0, S) = \{S' \to S \cdot \} \\
            I_2 = & GO(I_0, A) = \{S \to A \cdot aAb \} \\
            I_3 = & GO(I_0, B) = \{S \to B \cdot bBa \} \\
            \hline
            I_4 = & GO(I_2, a) = \{S \to Aa \cdot Ab, A \to \cdot\} \\
            \hline
            I_5 = & GO(I_3, b) = \{S \to Bb \cdot Ba, B \to \cdot\} \\
            \hline
            I_6 = & GO(I_4, A) = \{S \to AaA \cdot b\} \\
            \hline
            I_7 = & GO(I_5, B) = \{S \to BbB \cdot a\} \\
            \hline
            I_8 = & GO(I_6, b) = \{S \to AaAb \cdot\} \\
            \hline
            I_9 = & GO(I_7, a) = \{S \to BbBa \cdot\} \\
        \end{array}
    \end{equation*}
    
    构造 LR(0) 分析表:
    
    \begin{equation*}
        \begin{array}{c|ccc|ccc}
            I & a & b & \# & S & A & B \\
            \hline
            0 & r_4, r_5 & r_4, r_5 & r_4, r_5 & 1 & 2 & 3 \\
            1 & & & acc & & & \\
            2 & s_4 & & & & & \\
            3 & s_5 & & & & & \\
            4 & r_4 & r_4 & r_4 & & 6 & \\
            5 & r_5 & r_5 & r_5 & & & 7 \\
            6 & & s_8 & & & & \\
            7 & s_9 & & & & & \\
            8 & r_2 & r_2 & r_2 & & & \\
            9 & r_3 & r_3 & r_3 & & & \\
        \end{array}
    \end{equation*}
    
    在状态 0 处有规约-规约冲突,这表明该文法不是 LR(0) 文法。
    
    检查表 \ref{tab:FF_8} 得 FOLLOW(A) = FOLLOW(B),这使得 SLR 法无法解决冲突,因此\textbf{该文法不是 SLR(1) 文法}。
    
    综上所述,该文法是 LL(1) 的但不是 SLR(1) 的,\textbf{证毕}。
    
    \item[9.] 证明下面文法:
    
    \begin{align*}
        S & \to Aa | bAc | Bc | bBa \\
        A & \to d
    \end{align*}
    
    \sout{是 LALR(1) 但}不是 SLR(1) 的。
    
    \textbf{答:缺少对非终结符 B 的定义,此题不可做。}
    
    % \textbf{证:}
    
    % 拓广文法:
    
    % \begin{enumerate}[(1)]
    %     \item $S' \to S$
    %     \item $S \to Aa$
    %     \item $S \to bAc$
    %     \item $S \to Bc$
    %     \item $S \to bBa$
    %     \item $A \to d$
    % \end{enumerate}
    
    % LR(0) 项目集规范族:
    
    % \begin{equation*}
    %     \begin{array}{cl}
    %         I_0 = & \{S' \to \cdot S, S \to \cdot Aa, S \to \cdot bAc, S \to \cdot Bc, S \to \cdot bBa, A \to \cdot d\} \\
    %         \hline
    %         I_1 = & GO(I_0, S) = \{S' \to S \cdot\} \\ 
    %         I_2 = & GO(I_0, A) = \{S \to A \cdot a\} \\ 
    %         I_3 = & GO(I_0, b) = \{S \to b \cdot Ac, S \to b \cdot Ba, A \to \cdot d\} \\ 
    %         I_4 = & GO(I_0, B) = \{S \to B \cdot c\} \\ 
    %         I_5 = & GO(I_0, d) = \{A \to d \cdot\} \\ 
    %         \hline
    %         I_6 = & GO(I_2, a) = \{S \to Aa \cdot \} \\
    %         \hline
    %         I_7 = & GO(I_3, A) = \{S \to bA \cdot c\} \\
    %         I_8 = & GO(I_3, B) = \{S \to bB \cdot a\} \\
    %         I_9 = & GO(I_3, d) = \{A \to d \cdot\} \\
    %         \hline
    %         I_{10} = & GO(I_4, c) = \{S \to Bc \cdot\} \\
    %         \hline
    %         I_{11} = & GO(I_7, c) = \{S \to bAc \cdot\} \\
    %         \hline
    %         I_{12} = & GO(I_8, a) = \{S \to bBa \cdot\} \\
    %     \end{array}
    % \end{equation*}
    
\end{enumerate}