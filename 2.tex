% \usepackage{amsmath}
% \usepackage[normalem]{ulem}
\begin{enumerate}
    \item 考虑下述文法 $G_1$:
    
    \begin{align*}
        S & \to a | \wedge | (T) \\
        T & \to T,S | S
    \end{align*}
    
    \begin{enumerate}
        \item 消去 $G_1$ 的左递归。\sout{然后,对每个非终结符,写出不带回溯的递归子程序。}
        
        \textbf{答:}
        
        引入一个新的非终结符 $T'$:
        
        \begin{align*}
            T & \to ST' \\
            T' & \to ,ST' | \epsilon
        \end{align*}
        
        于是得到不含左递归的文法 $G_2$:
        
        \begin{align*}
            S & \to a | \wedge | (T) \\
            T & \to ST' \\
            T' & \to ,ST' | \epsilon
        \end{align*}
        
        \item 经改写后的文法是否是 LL(1) 的?给出它的预测分析表。
        
        \textbf{答:}
        
        首先算出文法 $G_2$ 的非终结符的 FIRST 集与 FOLLOW 集:
        
        \begin{table}[H]
            \centering
            \begin{tabular}{|c|c|c|}
                \hline
                $V_N$ & FIRST & FOLLOW \\
                \hline
                S & \{a, $\wedge$, (\} & \{\#, \textbf{,}, )\} \\
                \hline
                T & \{a, $\wedge$, (\} & \{)\} \\
                \hline
                T' & \{\textbf{,}, $\epsilon$\} & \{)\} \\
                \hline
            \end{tabular}
            \caption{$G_2$ 文法的非终结符的 FIRST 集与 FOLLOW 集}
            \label{tab:FF1}
        \end{table}
        
        然后据此构造预测分析表:
        
        \begin{table}[H]
            \centering
            \begin{tabular}{|c|c|c|c|c|c|c|}
                \hline
                $V_N$ & a & $\wedge$ & ( & ) & \textbf{,} & \# \\
                \hline
                S & $S \to a$ & $S \to \wedge$ & $S \to (T)$ & & & \\ 
                \hline 
                T & $T \to ST'$ & $T \to ST'$ & $T \to ST'$ & & & \\
                \hline
                T' & & & & $T' \to \epsilon$ & $T' \to ,ST$ & \\
                \hline
            \end{tabular}
            \caption{$G_2$ 文法的预测分析表}
            \label{tab:PA1}
        \end{table}
        
        根据表 \ref{tab:PA1} 得 $G_2$ 是 LL(1) 文法。
        
    \end{enumerate}
    
    \item 对下列的文法 G:
    
    \begin{align*}
        E & \to TE' \\
        E' & \to +E | \epsilon \\
        T & \to FT' \\
        T' & \to T | \epsilon \\
        F & \to PF' \\
        F' & \to *F' | \epsilon \\
        P & \to (E) | a | b | \wedge
    \end{align*}
    
    \begin{enumerate}
        \item 计算这个文法的每个非终结符的 FIRST 和 FOLLOW。
        
        \textbf{答:}
        
        \begin{table}[H]
            \centering
            \begin{tabular}{|c|c|c|}
                \hline
                $V_N$ & FIRST & FOLLOW \\
                \hline
                E & \{(, a, b, $\wedge$\} & \{\#, )\} \\
                \hline
                E' & \{+, $\epsilon$\}  & \{\#, )\} \\
                \hline
                T & \{(, a, b, $\wedge$\} & \{\#, ), +\} \\
                \hline
                T' & \{(, a, b, $\wedge$, $\epsilon$\} & \{\#, ), +\} \\
                \hline
                F & \{(, a, b, $\wedge$\} & \{\#, ), +, (, a, b, $\wedge$\} \\
                \hline
                F' & \{*, $\epsilon$\}  & \{\#, ), +, (, a, b, $\wedge$\} \\
                \hline
                P & \{(, a, b, $\wedge$\} & \{\#, ), +, (, a, b, $\wedge$, *\} \\
                \hline
            \end{tabular}
            \caption{文法 G 的非终结符的 FIRST 和 FOLLOW 集}
            \label{tab:FF2}
        \end{table}
        
        \item 证明这个文法是 LL(1) 的。
        
        \textbf{证:}
        
        文法 G 中,仅 E', T', F', P 具有多个候选式,分别讨论:
        
        \begin{enumerate}
            \item $E' \to +E | \epsilon$
            
            $FIRST(+E) \cap FIRST(\epsilon) = \{+\} \cap \{\epsilon\} = \emptyset$
            
            $FIRST(+E) \cap FOLLOW(E') = \{+\} \cap \{\#, )\} = \emptyset$
            
            \item $T' \to T | \epsilon$
            
            $FIRST(T) \cap FIRST(\epsilon) = \{(, a, b, \wedge\} \cap \{\epsilon\} = \emptyset$
            
            $FIRST(T) \cap FOLLOW(T') = \{(, a, b, \wedge\} \cap \{\#, ), +\} = \emptyset$
            
            \item $F' \to *F' | \epsilon$
            
            $FIRST(*F') \cap FIRST(\epsilon) = \{*\} \cap \{\epsilon\} = \emptyset$
            
            $FIRST(*F') \cap FOLLOW(F') = \{*\} \cap \{\#, ), +, (, a, b, \wedge\} = \emptyset$
            
            \item $P \to (E) | a | b | \wedge$
            
            $FIRST((E)) \cap (FIRST(a) \cup FIRST(b) \cup FIRST(\wedge)) = \emptyset$
            
            $FIRST(a) \cap (FIRST(b) \cup FIRST(\wedge)) = \emptyset$
            
            $FIRST(b) \cap FIRST(\wedge) = \emptyset$
            
            因此,$FIRST((E)), FIRST(a), FIRST(b), FIRST(\wedge)$ 两两不相交。
            
        \end{enumerate}
        
        综上所述,G 是 LL(1) 文法,\textbf{证毕。}
        
        \item 构造它的预测分析表。
        
        \begin{table}[H]
            \centering
            \begin{tabular}{|c|c|c|c|c|c|c|c|c|}
                \hline
                $V_N$ & a & b & + & * & $\wedge$ & ( & ) & \# \\
                \hline
                E & $E \to TE'$ & $E \to TE'$ & & & $E \to TE'$ & $E \to TE'$ & & \\
                \hline
                E' & & & $E' \to +E$ & & & & $E' \to \epsilon$ & $E' \to \epsilon$ \\
                \hline
                T & $T \to FT'$ & $T \to FT'$ & & & $T \to FT'$ & $T \to FT'$ & & \\
                \hline
                T' & $T' \to T$ & $T' \to T$ & $T \to \epsilon$ & & $T' \to T$ & $T' \to T$ & $T \to \epsilon$ & $T \to \epsilon$ \\
                \hline
                F & $F \to PF'$ & $F \to PF'$ & & & $F \to PF'$ & $F \to PF'$ & & \\
                \hline
                F' & $F' \to \epsilon$ & $F' \to \epsilon$ & $F' \to \epsilon$ & $F' \to *F'$ & $F' \to \epsilon$ & $F' \to \epsilon$ & $F' \to \epsilon$ & $F' \to \epsilon$ \\
                \hline
                P & $P \to a$ & $P \to b$ & & & $P \to \wedge$ & $P \to (E)$ & & \\
                \hline
            \end{tabular}
            \caption{文法 G 的预测分析表}
            \label{tab:PA2}
        \end{table}
        
        \item \sout{构造它的递归下降分析程序。}
    \end{enumerate}
    
    \item 下面文法中,哪些是 LL(1) 的,说明理由。
    
    \begin{enumerate}
        \item \begin{align*}
            S & \to ABc \\
            A & \to a | \epsilon \\
            B & \to b | \epsilon
        \end{align*}
        
        \textbf{答:是 LL(1) 文法。}
        
        \begin{table}[H]
            \centering
            \begin{tabular}{|c|c|c|}
                \hline
                $V_N$ & FIRST & FOLLOW \\
                \hline
                S & \{a, b, c\} & \{\#\} \\
                \hline
                A & \{a, $\epsilon$\} & \{b, c\} \\
                \hline
                B & \{b, $\epsilon$\} & \{c\} \\
                \hline
            \end{tabular}
            \caption{匿名文法的非终结符的 FIRST 和 FOLLOW 集}
            \label{tab:FF3_1}
        \end{table}
        
        \begin{enumerate}
            \item $A \to a | \epsilon$
            
            $FIRST(a) \cap FOLLOW(A) = \emptyset$
            
            \item $B \to b | \epsilon$
            
            $FIRST(b) \cap FOLLOW(B) = \emptyset$
            
        \end{enumerate}
        
        所以,该文法为 LL(1) 的。
        
        \item \begin{align*}
            S & \to Ab \\
            A & \to a | B | \epsilon \\
            B & \to b | \epsilon
        \end{align*}
        
        \textbf{答:不是 LL(1) 文法。}
        
        \begin{table}[H]
            \centering
            \begin{tabular}{|c|c|c|}
                \hline
                $V_N$ & FIRST & FOLLOW \\
                \hline
                S & \{a, b\} & \{\#\} \\
                \hline
                A & \{a, b, $\epsilon$\} & \{b\} \\
                \hline
                B & \{b, $\epsilon$\} & \{b\} \\
                \hline
            \end{tabular}
            \caption{匿名文法的非终结符的 FIRST 和 FOLLOW 集}
            \label{tab:FF3_2}
        \end{table}
        
        在 A 得到 b 时不能确定使用 $A \to B$ 还是 $A \to \epsilon$;
        
        在 B 得到 b 时不能确定使用 $B \to b$ 还是 $B \to \epsilon$。
        
        ($FIRST(B) \cap FOLLOW(A) = \{b\}, FIRST(b) \cap FOLLOW(B) = \{b\} $)
        
        \item \begin{align*}
            S & \to ABBA \\
            A & \to a | \epsilon \\
            B & \to b | \epsilon
        \end{align*}
        
        \textbf{答:是 LL(1) 文法。}
        
        \begin{table}[H]
            \centering
            \begin{tabular}{|c|c|c|}
                \hline
                $V_N$ & FIRST & FOLLOW \\
                \hline
                S & \{a, b, $\epsilon$\} & \{\#\} \\
                \hline
                A & \{a, $\epsilon$\} & \{\#, a, b\} \\
                \hline
                B & \{b, $\epsilon$\} & \{\#, a, b\} \\
                \hline
            \end{tabular}
            \caption{匿名文法的非终结符的 FIRST 和 FOLLOW 集}
            \label{tab:FF3_3}
        \end{table}
        
        \begin{enumerate}
            \item $A \to a | \epsilon$
            
            $FIRST(a) \cap FOLLOW(A) = \{a\}$
            
            \item $B \to b | \epsilon$
            
            $FIRST(b) \cap FOLLOW(B) = \{b\}$
            
        \end{enumerate}
        
        所以,该文法不是 LL(1) 的。
        
        \item \begin{align*}
            S & \to aSe | B \\
            B & \to bBe | C \\
            C & \to cCe | d
        \end{align*}
        
        \textbf{答:是 LL(1) 文法。}
        
        \begin{table}[H]
            \centering
            \begin{tabular}{|c|c|c|}
                \hline
                $V_N$ & FIRST & FOLLOW \\
                \hline
                S & \{a, b, c, d\} & \{\#, e\} \\
                \hline
                B & \{b, c, d\} & \{\#, e\} \\
                \hline
                C & \{c, d\} & \{\#, e\} \\
                \hline
            \end{tabular}
            \caption{匿名文法的非终结符的 FIRST 和 FOLLOW 集}
            \label{tab:FF3_4}
        \end{table}
        
        \begin{enumerate}
            \item $S \to aSe | B$
            
            $FIRST(aSe) \cap FIRST(B) = \emptyset$
            
            \item $B \to bBe | C$
            
            $FIRST(bBe) \cap FIRST(C) = \emptyset$
            
            \item $C \to cCe | d$
            
            $FIRST(cCe) \cap FIRST(d) = \emptyset$
            
        \end{enumerate}
        
        所以,该文法是 LL(1) 的。
        
    \end{enumerate}
    
    \item 对下面文法:
    
    \begin{align*}
        Expr & \to -Expr \\
        Expr & \to (Expr) | Var \text{\textvisiblespace} ExprTail \\
        ExprTail & \to -Expr | \epsilon \\
        Var & \to id \text{\textvisiblespace} VarTail \\
        VarTail & \to (Expr) | \epsilon
    \end{align*}
    
    \begin{enumerate}
        \item 构造 LL(1) 分析表。
        
        \textbf{答:}
                
        \begin{table}[H]
            \centering
            \begin{tabular}{|c|c|c|}
                \hline
                $V_N$ & FIRST & FOLLOW \\
                \hline
                Expr & \{id, -, (\} & \{\#, )\} \\
                \hline
                ExprTail & \{-, $\epsilon$\} & \{\#, )\} \\
                \hline
                Var & \{id\} & \{\textvisiblespace\} \\
                \hline
                VarTail & \{(, $\epsilon$\} & \{\textvisiblespace\} \\
                \hline
            \end{tabular}
            \caption{匿名文法的非终结符的 FIRST 和 FOLLOW 集}
            \label{tab:FF4}
        \end{table}
        
        \begin{table}[H]
            \centering
            \resizebox{\textwidth}{!}{
                \begin{tabular}{|c|c|c|c|c|c|c|}
                    \hline
                    $V_N$ & - & id & \textvisiblespace & ( & ) & \# \\
                    \hline
                    Expr & $Expr \to -Expr$ & $Expr \to Var \text{\textvisiblespace} ExprTail $ & & $Expr \to (Expr)$ & & \\
                    \hline
                    ExprTail & $ExprTail \to -Expr$ & & & & $ExprTail \to \epsilon$ & $ExprTail \to \epsilon$ \\
                    \hline
                    Var & & $Var \to id \text{\textvisiblespace} VarTail$ & & & & \\
                    \hline
                    VarTail & & & $VarTail \to \epsilon$ & $VarTail \to (Expr)$ & & \\
                    \hline
                \end{tabular}
            }
            \caption{匿名文法的 LL(1) 预测分析表}
            \label{tab:PA4}
        \end{table}
        
        \item \sout{给出对句子 id - - id((id)) 的分析过程。}
    \end{enumerate}
    
\end{enumerate}